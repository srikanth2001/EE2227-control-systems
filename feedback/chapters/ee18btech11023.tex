\begin{enumerate}[label=\thesection.\arabic*.,ref=\thesection.\theenumi]
\numberwithin{equation}{enumi}

\item
\label{Question_1_ee18btech11023}
The feedback current amplifier in fig.\ref{fig:Original ckt1} can
be thought of as a “super” CG transistor. Note that rather than
connecting the gate of $Q_2$ to signal ground, an amplifier is
placed between source and gate.
\begin{figure}[!ht]
	\begin{center}
			\resizebox{\columnwidth}{!}{\begin{circuitikz}
\ctikzset{bipoles/length=1cm}
\ctikzset{tripoles/mos style/arrows}
\draw 
(0, 0) node[op amp, yscale=-1] (opamp) {}
(opamp.+) -- (-1,0.35)node[ground,rotate=-90]{} 
(opamp.out) to (.5,-0)--(0.5,0)
(opamp.-) -- (opamp.-) -- (-0.6,-0.35) to[]  (-0.9,-0.35) ;
\draw (-0.9,-1)--(-0.9,-1)--(-1.2,-1)--(-1.2,-1.2) to[R,l_=$R_s$,*-*](-1.2,-2.5) to(-1.2,-2.6)node[ground](GND){};
\draw (-1.2,-1)--(-3,-1)--(-3,-1.5) to[isource, l= $I_{s}$] (-3,-2) node[ground]{};
\draw (opamp.center) node[]{$\mu$};
\draw (1.5,0) node[nmos,](Q2){};
\draw (Q2.S)--(1.5,-1);
\draw (1.5,-1)--(-0.9,-1)--(-0.9,-0.35);
\draw (1.5,0)(Q2.center) node[right]{{$Q_{2}$}};
\draw (Q2.G) -- (0.5,0) to [short] (0.5,0);
\draw (Q2.D) -- (1.5,0.8)  --(1.5,1.5)node[ground,rotate=180](GND){};
\draw (1.7,1.9)--(1.7,1.3)to[short, i=$I_o$](1.7,1.2);
\draw (3,0.8)--(1.7,0.8)--(1.7,0.6)to[short,i=$R_o$](1.7,0.4);
\draw (-0.8,-2.5)--(-0.8,-1.2)--(-0.5,-1.2)to[short,i=$.$](-0.4,-1.2)
node[right] at(-0.8,-2.5){$R_{in}$};
\end{circuitikz}
}
	\end{center}
\caption{}
\label{fig:Original ckt1}
\end{figure}
\\for the fig.\ref{fig:Original ckt1} , the parameter's  table is\;\;\;\;\; TABLE.\ref{table:ee18btech11023_1}
\begin{table}[]
    \centering
  	\resizebox{\columnwidth}{!}{%%%%%%%%%%%%%%%%%%%%%%%%%%%%%%%%%%%%%%%%%%%%%%%%%%%%%%%%%%%%%%%%%%%%%%
%%                                                                  %%
%%  This is the header of a LaTeX2e file exported from Gnumeric.    %%
%%                                                                  %%
%%  This file can be compiled as it stands or included in another   %%
%%  LaTeX document. The table is based on the longtable package so  %%
%%  the longtable options (headers, footers...) can be set in the   %%
%%  preamble section below (see PRAMBLE).                           %%
%%                                                                  %%
%%  To include the file in another, the following two lines must be %%
%%  in the including file:                                          %%
%%        \def\inputGnumericTable{}                                 %%
%%  at the beginning of the file and:                               %%
%%        \input{name-of-this-file.tex}                             %%
%%  where the table is to be placed. Note also that the including   %%
%%  file must use the following packages for the table to be        %%
%%  rendered correctly:                                             %%
%%    \usepackage[latin1]{inputenc}                                 %%
%%    \usepackage{color}                                            %%
%%    \usepackage{array}                                            %%
%%    \usepackage{longtable}                                        %%
%%    \usepackage{calc}                                             %%
%%    \usepackage{multirow}                                         %%
%%    \usepackage{hhline}                                           %%
%%    \usepackage{ifthen}                                           %%
%%  optionally (for landscape tables embedded in another document): %%
%%    \usepackage{lscape}                                           %%
%%                                                                  %%
%%%%%%%%%%%%%%%%%%%%%%%%%%%%%%%%%%%%%%%%%%%%%%%%%%%%%%%%%%%%%%%%%%%%%%



%%  This section checks if we are begin input into another file or  %%
%%  the file will be compiled alone. First use a macro taken from   %%
%%  the TeXbook ex 7.7 (suggestion of Han-Wen Nienhuys).            %%
\def\ifundefined#1{\expandafter\ifx\csname#1\endcsname\relax}


%%  Check for the \def token for inputed files. If it is not        %%
%%  defined, the file will be processed as a standalone and the     %%
%%  preamble will be used.                                          %%
\ifundefined{inputGnumericTable}

%%  We must be able to close or not the document at the end.        %%
	\def\gnumericTableEnd{\end{document}}


%%%%%%%%%%%%%%%%%%%%%%%%%%%%%%%%%%%%%%%%%%%%%%%%%%%%%%%%%%%%%%%%%%%%%%
%%                                                                  %%
%%  This is the PREAMBLE. Change these values to get the right      %%
%%  paper size and other niceties.                                  %%
%%                                                                  %%
%%%%%%%%%%%%%%%%%%%%%%%%%%%%%%%%%%%%%%%%%%%%%%%%%%%%%%%%%%%%%%%%%%%%%%

	\documentclass[12pt%
			  %,landscape%
                    ]{report}
       \usepackage[latin1]{inputenc}
       \usepackage{fullpage}
       \usepackage{color}
       \usepackage{array}
       \usepackage{longtable}
       \usepackage{calc}
       \usepackage{multirow}
       \usepackage{hhline}
       \usepackage{ifthen}



%%  End of the preamble for the standalone. The next section is for %%
%%  documents which are included into other LaTeX2e files.          %%
\else

%%  We are not a stand alone document. For a regular table, we will %%
%%  have no preamble and only define the closing to mean nothing.   %%
    \def\gnumericTableEnd{}

%%  If we want landscape mode in an embedded document, comment out  %%
%%  the line above and uncomment the two below. The table will      %%
%%  begin on a new page and run in landscape mode.                  %%
%       \def\gnumericTableEnd{\end{landscape}}
%       \begin{landscape}


%%  End of the else clause for this file being \input.              %%
\fi

%%%%%%%%%%%%%%%%%%%%%%%%%%%%%%%%%%%%%%%%%%%%%%%%%%%%%%%%%%%%%%%%%%%%%%
%%                                                                  %%
%%  The rest is the gnumeric table, except for the closing          %%
%%  statement. Changes below will alter the table's appearance.     %%
%%                                                                  %%
%%%%%%%%%%%%%%%%%%%%%%%%%%%%%%%%%%%%%%%%%%%%%%%%%%%%%%%%%%%%%%%%%%%%%%

\providecommand{\gnumericmathit}[1]{#1} 
%%  Uncomment the next line if you would like your numbers to be in %%
%%  italics if they are italizised in the gnumeric table.           %%
%\renewcommand{\gnumericmathit}[1]{\mathit{#1}}
\providecommand{\gnumericPB}[1]%
{\let\gnumericTemp=\\#1\let\\=\gnumericTemp\hspace{0pt}}
 \ifundefined{gnumericTableWidthDefined}
        \newlength{\gnumericTableWidth}
        \newlength{\gnumericTableWidthComplete}
        \newlength{\gnumericMultiRowLength}
        \global\def\gnumericTableWidthDefined{}
 \fi
%% The following setting protects this code from babel shorthands.  %%
 \ifthenelse{\isundefined{\languageshorthands}}{}{\languageshorthands{english}}
%%  The default table format retains the relative column widths of  %%
%%  gnumeric. They can easily be changed to c, r or l. In that case %%
%%  you may want to comment out the next line and uncomment the one %%
%%  thereafter                                                      %%
\providecommand\gnumbox{\makebox[0pt]}
%%\providecommand\gnumbox[1][]{\makebox}

%% to adjust positions in multirow situations                       %%
\setlength{\bigstrutjot}{\jot}
\setlength{\extrarowheight}{\doublerulesep}

%%  The \setlongtables command keeps column widths the same across  %%
%%  pages. Simply comment out next line for varying column widths.  %%
\setlongtables

\setlength\gnumericTableWidth{%
	53pt+%
	93pt+%
0pt}
\def\gumericNumCols{2}
\setlength\gnumericTableWidthComplete{\gnumericTableWidth+%
         \tabcolsep*\gumericNumCols*2+\arrayrulewidth*\gumericNumCols}
\ifthenelse{\lengthtest{\gnumericTableWidthComplete > \linewidth}}%
         {\def\gnumericScale{\ratio{\linewidth-%
                        \tabcolsep*\gumericNumCols*2-%
                        \arrayrulewidth*\gumericNumCols}%
{\gnumericTableWidth}}}%
{\def\gnumericScale{1}}

%%%%%%%%%%%%%%%%%%%%%%%%%%%%%%%%%%%%%%%%%%%%%%%%%%%%%%%%%%%%%%%%%%%%%%
%%                                                                  %%
%% The following are the widths of the various columns. We are      %%
%% defining them here because then they are easier to change.       %%
%% Depending on the cell formats we may use them more than once.    %%
%%                                                                  %%
%%%%%%%%%%%%%%%%%%%%%%%%%%%%%%%%%%%%%%%%%%%%%%%%%%%%%%%%%%%%%%%%%%%%%%

\ifthenelse{\isundefined{\gnumericColA}}{\newlength{\gnumericColA}}{}\settowidth{\gnumericColA}{\begin{tabular}{@{}p{180pt*\gnumericScale}@{}}x\end{tabular}}
\ifthenelse{\isundefined{\gnumericColB}}{\newlength{\gnumericColB}}{}\settowidth{\gnumericColB}{\begin{tabular}{@{}p{65pt*\gnumericScale}@{}}x\end{tabular}}

\begin{tabular}[c]{%
	b{\gnumericColA}%
	b{\gnumericColB}%
	}

%%%%%%%%%%%%%%%%%%%%%%%%%%%%%%%%%%%%%%%%%%%%%%%%%%%%%%%%%%%%%%%%%%%%%%
%%  The longtable options. (Caption, headers... see Goosens, p.124) %%
%	\caption{The Table Caption.}             \\	%
% \hline	% Across the top of the table.
%%  The rest of these options are table rows which are placed on    %%
%%  the first, last or every page. Use \multicolumn if you want.    %%

%%  Header for the first page.                                      %%
%	\multicolumn{2}{c}{The First Header} \\ \hline 
%	\multicolumn{1}{c}{colTag}	%Column 1
%	&\multicolumn{1}{c}{colTag}	\\ \hline %Last column
%	\endfirsthead

%%  The running header definition.                                  %%
%	\hline
%	\multicolumn{2}{l}{\ldots\small\slshape continued} \\ \hline
%	\multicolumn{1}{c}{colTag}	%Column 1
%	&\multicolumn{1}{c}{colTag}	\\ \hline %Last column
%	\endhead

%%  The running footer definition.                                  %%
%	\hline
%	\multicolumn{2}{r}{\small\slshape continued\ldots} \\
%	\endfoot

%%  The ending footer definition.                                   %%
%	\multicolumn{2}{c}{That's all folks} \\ \hline 
%	\endlastfoot
%%%%%%%%%%%%%%%%%%%%%%%%%%%%%%%%%%%%%%%%%%%%%%%%%%%%%%%%%%%%%%%%%%%%%%

\hhline{|-|-}
	 \multicolumn{1}{|p{\gnumericColA}|}%
	{\gnumericPB{\centering}\gnumbox{\textbf{Parameter}}}
	&\multicolumn{1}{p{\gnumericColB}|}%
	{\gnumericPB{\centering}\gnumbox{\textbf{Value}}}
\\
\hhline{|-|-}
	 \multicolumn{1}{|p{\gnumericColA}|}%
	{\gnumericPB{\centering}\gnumbox{\textbf{input resistance(large $\mu$)}}}
	&\multicolumn{1}{p{\gnumericColB}|}%
	{\gnumericPB{\centering}\gnumbox{\textbf{$0$}}}
\\
\hhline{|-|-}
	 \multicolumn{1}{|p{\gnumericColA}|}%
	{\gnumericPB{\centering}\gnumbox{\textbf{output resistance(large $\mu$)}}}
	&\multicolumn{1}{p{\gnumericColB}|}%
	{\gnumericPB{\centering}\gnumbox{\textbf{$\infty$}}}
	\\
\hhline{|-|-}
	 \multicolumn{1}{|p{\gnumericColA}|}%
	{\gnumericPB{\centering}\gnumbox{\textbf{Input voltage}}}
	&\multicolumn{1}{p{\gnumericColB}|}%
	{\gnumericPB{\centering}\gnumbox{\textbf{$- I_s R_s$}}}

\\
\hhline{|-|-}
	 \multicolumn{1}{|p{\gnumericColA}|}%
	{\gnumericPB{\centering}\gnumbox{\textbf{input resistance(finite $\mu$)}}}
	&\multicolumn{1}{p{\gnumericColB}|}%
	{\gnumericPB{\centering}\gnumbox{\textbf{$R_s$}}}
\\
\hhline{|-|-}
	 \multicolumn{1}{|p{\gnumericColA}|}%
	{\gnumericPB{\centering}\gnumbox{\textbf{output resistance(finite  $\mu$)}}}
	&\multicolumn{1}{p{\gnumericColB}|}%
	{\gnumericPB{\centering}\gnumbox{\textbf{$r_o$}}}


\\

\hhline{|-|-}
	 \multicolumn{1}{|p{\gnumericColA}|}%
	{\gnumericPB{\centering}\gnumbox{\textbf{source resistance}}}
	&\multicolumn{1}{p{\gnumericColB}|}%
	{\gnumericPB{\centering}\gnumbox{\textbf{$R_s$}}}

\\
\hhline{|-|-}
	 \multicolumn{1}{|p{\gnumericColA}|}%
	{\gnumericPB{\centering}\gnumbox{\textbf{feedback factor H}}}
	&\multicolumn{1}{p{\gnumericColB}|}%
	{\gnumericPB{\centering}\gnumbox{\textbf{1}}}
\\
\hhline{|-|-}
	 \multicolumn{1}{|p{\gnumericColA}|}%
	{\gnumericPB{\centering}\gnumbox{\textbf{Open Loop Gain, G}}}
	&\multicolumn{1}{p{\gnumericColB}|}%
	{\gnumericPB{\centering}\gnumbox{$\mu g_m R_s$ $A/A$}}
\\
\hhline{|-|-}
	 \multicolumn{1}{|p{\gnumericColA}|}%
	{\gnumericPB{\centering}\gnumbox{\textbf{Closed Loop Gain, T}}}
	&\multicolumn{1}{p{\gnumericColB}|}%
	{\gnumericPB{\centering}\gnumbox{\textbf{1 A/A}}}
\\
\hhline{|-|-|}
\end{tabular}
\ifthenelse{\isundefined{\languageshorthands}}{}{\languageshorthands{\languagename}}
\gnumericTableEnd
}
    \caption{}
    \label{table:ee18btech11023_1}
\end{table}

\item 
\label{Question_1a_ee18btech11023}
(a) If $\mu$ is very large, what is the signal voltage at the input
terminal?What is the input resistance?What is the current
gain Io/Is ?\\
\\
\solution \\
Refer to the fig. \ref{fig:Original ckt1} for the feedback current amplifier circuit, in this super common gate transistor is connected between the gate and source terminals of the MOSFET.\\
Replace the op-amp with its equivalent modal and replace the MOSFET with its small signal equivalent circuit.\\
with reference to the fig,\ref{fig:small ckt}. For ideal op-amp,the input resistance($R_{id}$) is very high (infinite) . And the drain current is approximately equal to source current
\begin{align}
    I_{D} \cong  I_{S}
\label{eq_ee18btech11023_1}
\end{align}
The closed loop gain of op-amp is
\begin{align}
    T = \frac{\mu}{1 + \mu H}
    \label{eq_ee18btech11023_2}
\end{align}
for larger value of closed loop gain , open loop gain '$\mu$' will be large. 
from fig.\ref{fig:small ckt} we can observe that input voltage is,
\begin{align}
    V_{in} = -R_sI_s
    \label{eq_ee18btech11023_3}
\end{align}
Since the drain current is approximately is equal to source current, And the current flowing through resistor $R_S$ is $I_s$ since there is no current flowing through the negative terminal of  op-amp. Therefore, it is connected that the output current $I_O$ is flowing through the resister $R_s$, then
\begin{align}
    I_o = I_S
    \label{eq_ee18btech11023_4}
\end{align}
\begin{align*}
    \frac{I_o}{I_s} \equiv 1 
\end{align*}
therefore the current gain is,
\begin{align}
    \frac{I_o}{I_s} = 1 \; A/A
    \label{eq_ee18btech11023_5}
\end{align}
and from fig.\ref{fig:small ckt} the input Resistance is equal to $R_{id}$
\begin{align}
    R_{i} = R_{id}
    \label{eq_ee18btech11023_6}
\end{align}

\item
\label{Question_1b_ee18btech11023}
(b) For finite $\mu$ but assuming that the input resistance of the
amplifier g is very large, find the 'G' circuit and derive
expressions for G, $Ri$, and $R_o$.?
\\
\solution\\
For the finite value of $\mu$ and the input resistance of a ideal op-amp is very high(infinite).
the open loop amplifier circuit ('G' circuit ) fig.\ref{fig:small ckt}
\begin{figure}[!ht]
	\begin{center}
			\resizebox{\columnwidth}{!}{\begin{circuitikz}
\ctikzset{bipoles/length=1cm}

\draw (0.5, 1) to[isource, l= $I_{s}$](0.5,-1.5)--(0.5,-1.5)node[ground]{};
\draw (0.5,1)--(4,1);
\draw (2,1)--(2,0);
\draw node at (2,1.2){$V_{id}$};
\draw node at (4.1,-0.4){$R_{id\;\;( \infty)}$};
\draw (4,-0.8)--(4,-1.5)node[ground]{};
\draw (2,0) to[R,l_=$R_s$](2,-1)--(2,-1.5)node[ground]{};
\draw (4,1)--(4,0);
\draw node at (2.5,0.8){$-$};
\draw node at (2.5,-1.5){$+$};
\draw node at (2.5,-0.5){$V_{id}$};
\draw (6,1)--(6.5,1) to[R,l_=$r_o$](7.5,1) to[short,-o](8,1)--(8.6,1);
\draw (6,1) to[cisource, l= $\mu V_{id}$](6,-1.5)node[ground]{};
\draw (8.6,1)--(8.6,0)to[R,l_=$\frac{1}{g_m}$](8.6,-1) to[short,-o](8.6,-1.4)--(8.6,-1.5)node[ground]{};
\draw (10.5,1) to[cisource](8.6,1);
\draw (10.5,1)to[short,-o](10.6,1)node[ground,rotate=90]{};
\draw (9,-0.5)--(9,-1) to[short,i=$I_{i}$] (9,-1.1);
\draw (9.9,0.4)--(9.3,0.4) to[short,i=$I_o$](9.25,0.4);
\draw (10.7,-1.5)--(10.7,0.4)--(10.4,0.4) to[short,i=$.$](10.35,0.4);
\draw node at (10.4,-1.3){$R_o$};
\draw node at (8,1.3){$G$};
\draw node at (10.6,1.3){$D$};
\draw node at (8.9,-1.4){$S$};
\draw (3.3,-1.5)--(3.3,0.5)--(3.6,0.5) to[short,i=$.$](3.8,0.5);
\draw node at (3,-1.3){$R_{in}$};
\end{circuitikz}
}
	\end{center}
\caption{}
\label{fig:small ckt}
\end{figure}\\

For a ideal op-amp the output Resistance is very small, so we can neglect the resistance $r_o$.  from fig\ref{fig:small ckt}
\begin{align}
    V_{id} = I_sR_s
    \label{eq_ee18btech11023_7}
\end{align}
current through resistance $R_s$ is
\begin{align*}
    since\;\;I_o \equiv I_s
\end{align*}
\begin{align}
    I_o = \frac{\mu V_{id}}{\frac{1}{g_m}} 
    \label{eq_ee18btech11023_8}
\end{align}
\begin{align}
    I_o = \mu g_m I_s R_S
    \label{eq_ee18btech11023_9}
\end{align}
Expression for current gain G is:
\begin{align}
    \frac{I_o}{I_s} = \mu g_m R_s
    \label{eq_ee18btech11023_10}
\end{align}
\begin{align}
    G = \mu g_m R_s
    \label{eq_ee18btech11023_11}
\end{align}
from the fig.\ref{fig:small ckt}
\begin{align*}
    since \;\; R_{id} = \infty
\end{align*}
the input resistance:
\begin{align}
    R_I = R_s
    \label{eq_ee18btech11023_12}
\end{align}
the output Resistance:
\begin{align}
    R_o = r_o
    \label{eq_ee18btech11023_13}
\end{align}

\item
\label{Question_1c_ee18btech11023}
(c) What is the value of H?
\\
\solution\\
closed loop gain of the op-amp is:
\begin{align}
    T = \frac{\mu}{1 + \mu H}
    \label{eq_ee18btech11023_14}
\end{align}
\begin{align*}
    since \;\; \mu H >> 1
\end{align*}
then 
\begin{align}
    T = \frac{\mu}{\mu H}
    \label{eq_ee18btech11023_15}
\end{align}
for the larger value of $\mu$
\begin{align*}
   T \implies 1 
\end{align*}

the value of 'H' will be:
\begin{align}
    H = 1
    \label{eq_ee18btech11023_16}
\end{align}
\item
\label{Question_1d_ee18btech11023}
(d) Find GH and T . If $\mu$ is large, what is the value of T ?\\
\solution\\
from eq.\ref{eq_ee18btech11023_10} and from eq.\ref{eq_ee18btech11023_16}
\begin{align*}
    G = \mu g_m R_s\;\; ,  \;\;\; H = 1
\end{align*}
value of GH:
\begin{align}
    GH = \mu g_m R_s
     \label{eq_ee18btech11023_17}
\end{align}
closed loop gain of op-amp is:
\begin{align}
    T = \frac{G}{1+GH}
     \label{eq_ee18btech11023_18}
\end{align}
substitute values of G and GH in eq.\ref{eq_ee18btech11023_18}
\begin{align}
    T = \frac{\mu g_m R_s}{1+\mu g_m R_s}
    \label{eq_ee18btech11023_19}
\end{align}
for the larger valve of $\mu$
\begin{align}
    T \implies 1
    \label{eq_ee18btech11023_20}
\end{align}

\item
\label{Question_1e_ee18btech11023}
(e) Find $R_{in}$ and $R_{out}$ assuming the loop gain is large\\
\solution\\
\textbf{for a Feedback Amplifier }
\begin{align}
    R_{if} = \frac{R_{i}}{1+GH}
    \label{eq_ee18btech11023_21}
\end{align}
Here,
    $R_{i}$ =  $R_{s}$\\
substituting the values of GH and $R_{i}$ in eq.\ref{eq_ee18btech11023_21}\;\;\; we get:
\begin{align}
    R_{if} = \frac{R_s}{1+\mu g_m R_s}
    \label{eq_ee18btech11023_22}
\end{align}
dividing the eq.\ref{eq_ee18btech11023_22} with $R_s$
\begin{align}
    R_{if} = \frac{1}{\frac{1}{R_s} + \mu g_m}
    \label{eq_ee18btech11023_23}
\end{align}
the eq.\ref{eq_ee18btech11023_23} can also written like
\begin{align}
    R_{in} = R_{s} \parallel  \frac{1}{\mu g_m}
    \label{eq_ee18btech11023_24}
\end{align}
\begin{align}
    R_{if} = R_s \parallel R_{in}
    \label{eq_ee18btech11023_25}
\end{align}
 from the equations \ref{eq_ee18btech11023_24} and \ref{eq_ee18btech11023_25} $R_{in}$ ca be written as:

\begin{align}
    R_{in} = \frac{1}{\mu g_m}
    \label{eq_ee18btech11023_26}
\end{align}
for the lager value of $\mu$ then $R_{}$ will be small
\begin{align*}
    for\;\;\; \mu \implies \infty
\end{align*}
then $R_{in}$ becomes:
\begin{align}
    R_{in} = 0
    \label{eq_ee18btech11023_27}
\end{align}
\textbf{for a Feedback Amplifier output Resistance is: }
\begin{align}
    R_{out} = (1 + GH)R_o
    \label{eq_ee18btech11023_28}
\end{align}
From the figure Fig.\ref{fig:small ckt} 
\begin{align}
    R_o = \frac{1}{g_m}
    \label{eq_ee18btech11023_29}
\end{align}
substitute the values of GH and $R_o$ in eq.\ref{eq_ee18btech11023_28}\\
we get:
\begin{align}
    R_{out} = (1+ \mu g_m R_s)\frac{1}{g_m}
    \label{eq_ee18btech11023_30}
\end{align}
\begin{align}
    R_{out} = \frac{1}{g_m} + \mu R_s
    \label{eq_ee18btech11023_31}
\end{align}
By observing the eq.\ref{eq_ee18btech11023_31} , for the lager value for $\mu$ we will have larger value for $R_{out}$
\begin{align*}
    \mu \implies \infty
\end{align*}
\begin{align}
    R_{out} \implies \infty
    \label{eq_ee18btech11023_32}
\end{align}
\newpage
\item
\label{Question_2f_ee18btech11023}
(f) The “super” CG transistor can be utilized in the cascode
configuration shown in Fig.\ref{fig:Original ckt2}, where $V_G$ is a dc
bias voltage. Replacing Q1 by its small-signal model, use
the analogy of the resulting circuit to that in Fig.\ref{fig:Original ckt1}
to find Io and Rout.

\textbf{\begin{figure}[!ht]
	\begin{center}
			\resizebox{\columnwidth}{!}{\begin{circuitikz}
\ctikzset{bipoles/length=1cm}
\ctikzset{tripoles/mos style/arrows}
\draw 
(0, 0) node[op amp, yscale=-1] (opamp) {}
(opamp.+) -- (-1,0.35)to[short,-o](-1,0.35) 
node at (-1.3,0.35){$V_G$}
(opamp.-) 
(opamp.out) to (.5,-0)--(0.5,0)
(opamp.-) -- (-0.6,-0.35) to[]  (-0.9,-0.35) ;
\draw (1.5,0) node[nmos,](Q2){};
\draw (Q2.S)--(1.5,-1.5);
\draw  (1.5,-1)--(-0.9,-1)--(-0.9,-0.35);
\draw (1.5,0)(Q2.center) node[right]{{$Q_{2}$}};
\draw (Q2.G) -- (0.5,0) to [short] (0.5,0);
\draw (Q2.D) -- (1.5,1.5)node[ground,rotate=180](GND){};
\draw (1.7,1.7)--(1.7,1.3)to[short, i=$I_O$](1.7,1.1);
\draw (1.5,-2) node[nmos,](Q1){};
\draw (Q1.S)--(1.5,-3)node[ground](GND){};
\draw (Q1.center) node[right]{{$Q_{1}$}};
\draw (Q1.G) --(0,-2);
\draw (Q1.D) -- (1.5,-1.5);
\draw (-0,-2)to[short,-o](-0,-2)
node at(-0.3,-2){$V_i$};
\draw (3,0.8)--(1.7,0.8)--(1.7,0.6)to[short,i=$R_o$](1.7,0.4);
\end{circuitikz}
}
	\end{center}
\caption{}
\label{fig:Original ckt2}
\end{figure}}
\\for the fig.\ref{fig:Original ckt2} , the parameter's  table is\;\;\;\;\;
TABLE.\ref{table:ee18btech11023_2}
\begin{table}[]
    \centering
  	\resizebox{\columnwidth}{!}{%%%%%%%%%%%%%%%%%%%%%%%%%%%%%%%%%%%%%%%%%%%%%%%%%%%%%%%%%%%%%%%%%%%%%%
%%                                                                  %%
%%  This is the header of a LaTeX2e file exported from Gnumeric.    %%
%%                                                                  %%
%%  This file can be compiled as it stands or included in another   %%
%%  LaTeX document. The table is based on the longtable package so  %%
%%  the longtable options (headers, footers...) can be set in the   %%
%%  preamble section below (see PRAMBLE).                           %%
%%                                                                  %%
%%  To include the file in another, the following two lines must be %%
%%  in the including file:                                          %%
%%        \def\inputGnumericTable{}                                 %%
%%  at the beginning of the file and:                               %%
%%        \input{name-of-this-file.tex}                             %%
%%  where the table is to be placed. Note also that the including   %%
%%  file must use the following packages for the table to be        %%
%%  rendered correctly:                                             %%
%%    \usepackage[latin1]{inputenc}                                 %%
%%    \usepackage{color}                                            %%
%%    \usepackage{array}                                            %%
%%    \usepackage{longtable}                                        %%
%%    \usepackage{calc}                                             %%
%%    \usepackage{multirow}                                         %%
%%    \usepackage{hhline}                                           %%
%%    \usepackage{ifthen}                                           %%
%%  optionally (for landscape tables embedded in another document): %%
%%    \usepackage{lscape}                                           %%
%%                                                                  %%
%%%%%%%%%%%%%%%%%%%%%%%%%%%%%%%%%%%%%%%%%%%%%%%%%%%%%%%%%%%%%%%%%%%%%%



%%  This section checks if we are begin input into another file or  %%
%%  the file will be compiled alone. First use a macro taken from   %%
%%  the TeXbook ex 7.7 (suggestion of Han-Wen Nienhuys).            %%
\def\ifundefined#1{\expandafter\ifx\csname#1\endcsname\relax}


%%  Check for the \def token for inputed files. If it is not        %%
%%  defined, the file will be processed as a standalone and the     %%
%%  preamble will be used.                                          %%
\ifundefined{inputGnumericTable}

%%  We must be able to close or not the document at the end.        %%
	\def\gnumericTableEnd{\end{document}}


%%%%%%%%%%%%%%%%%%%%%%%%%%%%%%%%%%%%%%%%%%%%%%%%%%%%%%%%%%%%%%%%%%%%%%
%%                                                                  %%
%%  This is the PREAMBLE. Change these values to get the right      %%
%%  paper size and other niceties.                                  %%
%%                                                                  %%
%%%%%%%%%%%%%%%%%%%%%%%%%%%%%%%%%%%%%%%%%%%%%%%%%%%%%%%%%%%%%%%%%%%%%%

	\documentclass[12pt%
			  %,landscape%
                    ]{report}
       \usepackage[latin1]{inputenc}
       \usepackage{fullpage}
       \usepackage{color}
       \usepackage{array}
       \usepackage{longtable}
       \usepackage{calc}
       \usepackage{multirow}
       \usepackage{hhline}
       \usepackage{ifthen}



%%  End of the preamble for the standalone. The next section is for %%
%%  documents which are included into other LaTeX2e files.          %%
\else

%%  We are not a stand alone document. For a regular table, we will %%
%%  have no preamble and only define the closing to mean nothing.   %%
    \def\gnumericTableEnd{}

%%  If we want landscape mode in an embedded document, comment out  %%
%%  the line above and uncomment the two below. The table will      %%
%%  begin on a new page and run in landscape mode.                  %%
%       \def\gnumericTableEnd{\end{landscape}}
%       \begin{landscape}


%%  End of the else clause for this file being \input.              %%
\fi

%%%%%%%%%%%%%%%%%%%%%%%%%%%%%%%%%%%%%%%%%%%%%%%%%%%%%%%%%%%%%%%%%%%%%%
%%                                                                  %%
%%  The rest is the gnumeric table, except for the closing          %%
%%  statement. Changes below will alter the table's appearance.     %%
%%                                                                  %%
%%%%%%%%%%%%%%%%%%%%%%%%%%%%%%%%%%%%%%%%%%%%%%%%%%%%%%%%%%%%%%%%%%%%%%

\providecommand{\gnumericmathit}[1]{#1} 
%%  Uncomment the next line if you would like your numbers to be in %%
%%  italics if they are italizised in the gnumeric table.           %%
%\renewcommand{\gnumericmathit}[1]{\mathit{#1}}
\providecommand{\gnumericPB}[1]%
{\let\gnumericTemp=\\#1\let\\=\gnumericTemp\hspace{0pt}}
 \ifundefined{gnumericTableWidthDefined}
        \newlength{\gnumericTableWidth}
        \newlength{\gnumericTableWidthComplete}
        \newlength{\gnumericMultiRowLength}
        \global\def\gnumericTableWidthDefined{}
 \fi
%% The following setting protects this code from babel shorthands.  %%
 \ifthenelse{\isundefined{\languageshorthands}}{}{\languageshorthands{english}}
%%  The default table format retains the relative column widths of  %%
%%  gnumeric. They can easily be changed to c, r or l. In that case %%
%%  you may want to comment out the next line and uncomment the one %%
%%  thereafter                                                      %%
\providecommand\gnumbox{\makebox[0pt]}
%%\providecommand\gnumbox[1][]{\makebox}

%% to adjust positions in multirow situations                       %%
\setlength{\bigstrutjot}{\jot}
\setlength{\extrarowheight}{\doublerulesep}

%%  The \setlongtables command keeps column widths the same across  %%
%%  pages. Simply comment out next line for varying column widths.  %%
\setlongtables

\setlength\gnumericTableWidth{%
	53pt+%
	93pt+%
0pt}
\def\gumericNumCols{2}
\setlength\gnumericTableWidthComplete{\gnumericTableWidth+%
         \tabcolsep*\gumericNumCols*2+\arrayrulewidth*\gumericNumCols}
\ifthenelse{\lengthtest{\gnumericTableWidthComplete > \linewidth}}%
         {\def\gnumericScale{\ratio{\linewidth-%
                        \tabcolsep*\gumericNumCols*2-%
                        \arrayrulewidth*\gumericNumCols}%
{\gnumericTableWidth}}}%
{\def\gnumericScale{1}}

%%%%%%%%%%%%%%%%%%%%%%%%%%%%%%%%%%%%%%%%%%%%%%%%%%%%%%%%%%%%%%%%%%%%%%
%%                                                                  %%
%% The following are the widths of the various columns. We are      %%
%% defining them here because then they are easier to change.       %%
%% Depending on the cell formats we may use them more than once.    %%
%%                                                                  %%
%%%%%%%%%%%%%%%%%%%%%%%%%%%%%%%%%%%%%%%%%%%%%%%%%%%%%%%%%%%%%%%%%%%%%%

\ifthenelse{\isundefined{\gnumericColA}}{\newlength{\gnumericColA}}{}\settowidth{\gnumericColA}{\begin{tabular}{@{}p{160pt*\gnumericScale}@{}}x\end{tabular}}
\ifthenelse{\isundefined{\gnumericColB}}{\newlength{\gnumericColB}}{}\settowidth{\gnumericColB}{\begin{tabular}{@{}p{60pt*\gnumericScale}@{}}x\end{tabular}}

\begin{tabular}[c]{%
	b{\gnumericColA}%
	b{\gnumericColB}%
	}

%%%%%%%%%%%%%%%%%%%%%%%%%%%%%%%%%%%%%%%%%%%%%%%%%%%%%%%%%%%%%%%%%%%%%%
%%  The longtable options. (Caption, headers... see Goosens, p.124) %%
%	\caption{The Table Caption.}             \\	%
% \hline	% Across the top of the table.
%%  The rest of these options are table rows which are placed on    %%
%%  the first, last or every page. Use \multicolumn if you want.    %%

%%  Header for the first page.                                      %%
%	\multicolumn{2}{c}{The First Header} \\ \hline 
%	\multicolumn{1}{c}{colTag}	%Column 1
%	&\multicolumn{1}{c}{colTag}	\\ \hline %Last column
%	\endfirsthead

%%  The running header definition.                                  %%
%	\hline
%	\multicolumn{2}{l}{\ldots\small\slshape continued} \\ \hline
%	\multicolumn{1}{c}{colTag}	%Column 1
%	&\multicolumn{1}{c}{colTag}	\\ \hline %Last column
%	\endhead

%%  The running footer definition.                                  %%
%	\hline
%	\multicolumn{2}{r}{\small\slshape continued\ldots} \\
%	\endfoot

%%  The ending footer definition.                                   %%
%	\multicolumn{2}{c}{That's all folks} \\ \hline 
%	\endlastfoot
%%%%%%%%%%%%%%%%%%%%%%%%%%%%%%%%%%%%%%%%%%%%%%%%%%%%%%%%%%%%%%%%%%%%%%

\hhline{|-|-}
	 \multicolumn{1}{|p{\gnumericColA}|}%
	{\gnumericPB{\centering}\gnumbox{\textbf{Parameter}}}
	&\multicolumn{1}{p{\gnumericColB}|}%
	{\gnumericPB{\centering}\gnumbox{\textbf{Value}}}

\\
\hhline{|-|-}
	 \multicolumn{1}{|p{\gnumericColA}|}%
	{\gnumericPB{\centering}\gnumbox{\textbf{output resistance(large $\mu$)}}}
	&\multicolumn{1}{p{\gnumericColB}|}%
	{\gnumericPB{\centering}\gnumbox{\textbf{$\mu g_{m2}R_s R_o$}}}
	\\
\hhline{|-|-}
	 \multicolumn{1}{|p{\gnumericColA}|}%
	{\gnumericPB{\centering}\gnumbox{\textbf{Input voltage}}}
	&\multicolumn{1}{p{\gnumericColB}|}%
	{\gnumericPB{\centering}\gnumbox{\textbf{$ I_s/ g_{m1}$}}}



\\

\hhline{|-|-}
	 \multicolumn{1}{|p{\gnumericColA}|}%
	{\gnumericPB{\centering}\gnumbox{\textbf{output current $I_o$}}}
	&\multicolumn{1}{p{\gnumericColB}|}%
	{\gnumericPB{\centering}\gnumbox{\textbf{$g_{m1}V_i$}}}

\\
\hhline{|-|-}
	 \multicolumn{1}{|p{\gnumericColA}|}%
	{\gnumericPB{\centering}\gnumbox{\textbf{feedback factor H}}}
	&\multicolumn{1}{p{\gnumericColB}|}%
	{\gnumericPB{\centering}\gnumbox{\textbf{1}}}
\\
\hhline{|-|-}
	 \multicolumn{1}{|p{\gnumericColA}|}%
	{\gnumericPB{\centering}\gnumbox{\textbf{Open Loop Gain, G}}}
	&\multicolumn{1}{p{\gnumericColB}|}%
	{\gnumericPB{\centering}\gnumbox{$\mu g_{m2} R_s$ $A/A$}}
\\
\hhline{|-|-}
	 \multicolumn{1}{|p{\gnumericColA}|}%
	{\gnumericPB{\centering}\gnumbox{\textbf{Closed Loop Gain, T}}}
	&\multicolumn{1}{p{\gnumericColB}|}%
	{\gnumericPB{\centering}\gnumbox{\textbf{1 A/A}}}
\\
\hhline{|-|-|}
\end{tabular}
\ifthenelse{\isundefined{\languageshorthands}}{}{\languageshorthands{\languagename}}
\gnumericTableEnd
}
    \caption{}
    \label{table:ee18btech11023_2}
\end{table}
\item
\solution\\
Refer to the fig.\ref{fig:Original ckt2} for the cascode configuration in which the super CG transistor is used.  the small signal equivalent circuit is shown in 
fig.\ref{fig:small ckt2},

\begin{figure}[!ht]
	\begin{center}
			\resizebox{\columnwidth}{!}{\begin{circuitikz}
\ctikzset{bipoles/length=1cm}


\draw node at(0.25,-2.5){$V_{gs} - V_i$};
\draw node at(7.7,-2.5){$V_{id} - V_G$};


\draw node at (3.3,2){$small\;\; signal\;\;  of\;\; Q_1$};
\draw node at (8.1,2){$\;\;circuit  $};
\draw node at (8.1,2.5){$op-amp\; \;equivalent $};
\draw (0,1)--(0,0.6);
\draw (0,-1.4)--(0,-1.8)node[ground]{};
\draw node at (0,0.3){$+$};
\draw node at (0,-1.3){$-$};
\draw node at (0,-0.5){$V_i$};
\draw (0,1)--(1.3,1) to[short,-o](1.4,1)-- (2,1);
\draw node at (1.3,1.2){$G$};
\draw (2,1) --(2,0.6);
\draw (2,-1.4)--(2,-1.8)node[ground]{};
\draw node at (2,0.3){$+$};
\draw node at (2,-1.3){$-$};
\draw node at (2,-0.5){$V_{gs}$};
\draw (3.3,1)to[short,-o](5.5,1)--(5.5,-0.4)to[R,l_=$r_{o1}$,*-*](5.5,-1.4)--(5.5,-1.5)to[short,-o](5.5,-1.7)--(5.5,-1.8)node[ground]{};
\draw node at (5.5,1.2){$D$};
\draw (3.3,1)to[cisource, l= $g_{m1}V_{gs}$]( 3.3,-1.8)node[ground]{};
\draw (5.5,1)--(7,1)--(7,0.2)to[R,l_=$R_{id}$,*-*](7,-0.8)--(7,-1.8)node[ground]{};
\draw node at (7.5,0.5){$-$};
\draw node at (7.5,-0.4){$V_{id}$};
\draw node at (7.5,-1.3){$+$};
\draw (8.5,1)--(9,1) to[R,l_=$r_{o}$,*-*](10,1)--(10.5,1)to[short,-o](10.5,1)--(11,1);
\draw node at(5.3,-1.7){$S$};
\draw node at (10.6,1.2){$G$};
\draw node at (10.8,-1.7){$S$};
\draw node at (13,1.2){$D$};
\draw (8.5,1)to[cisource, l=$g_{m2}V_{id}$](8.5,-1.8)node[ground]{};
\draw (11,1)--(11,-0.4)to[R,l_=$r_{o1}$,*-*](11,-1.4)--(11,-1.5)to[short,-o](11,-1.7)--(11,-1.8)node[ground]{};
\draw (13.1,1)to[cisource, l= $.$](11,1);
\draw node at (12.9,-1.5){$R_O$};
\draw (13.1,-1.8)--(13.1,0.4)--(12.8,0.4)to[short,i=$.$](12.5,0.4);
\draw (12.5,1.6)--(11.6,1.6)to[short,i=$I_o$](11.5,1.6);
\draw (11.4,-0.4)--(11.4,-1)to[short,i=$I_{s}$] (11.4,-1.4);
\draw (13.1,1)to[short,-o](13.1,1) node[ground,rotate=90]{};
\draw[dashed] (1,1.6) -- (6,1.6) -- (6,-3) -- (1,-3) -- (1,1.6);
\draw[dashed] (6.2,1.6) -- (10,1.6) -- (10,-3) -- (6.2,-3) -- (6.2,1.6);

\end{circuitikz}}
	\end{center}
\caption{}
\label{fig:small ckt2}
\end{figure}

from the fig.\ref{fig:small ckt2} clearly observed that the current $I_s$
\begin{align}
    I_s = g_{m1} V_{gs}
    \label{eq_ee18btech11023_33}
\end{align}
the voltage $V_i$ is applied at the gate terminal of the transistor $Q_1$ ,
therefore the gate source voltage becomes $V_i$\;:
\begin{align}
    V_i = V_{gs}
    \label{eq_ee18btech11023_34}
\end{align}

form the eq.\ref{eq_ee18btech11023_34} and eq.\ref{eq_ee18btech11023_33}
\begin{align}
    I_s = g_{m1} V_i
    \label{eq_ee18btech11023_35}
\end{align}
the closed loop gain of the op-amp is:
\begin{align}
    T = \frac{G}{1 + GH}
    \label{eq_ee18btech11023_36}
\end{align}
froms the questions  Q.\ref{Question_1b_ee18btech11023} and Q.\ref{Question_1c_ee18btech11023}
 we will get:
\begin{align}
    G = \mu g_{m2}R_s
    \label{eq_ee18btech11023_37}
\end{align}
\begin{align}
    H \implies 1
    \label{eq_ee18btech11023_38}
\end{align}
substitute the values of H and G in eq.\ref{eq_ee18btech11023_36}
\begin{align}
    T = \frac{\mu g_{m2}R_s}{1 + \mu g_{m2}R_s}
    \label{eq_ee18btech11023_39}
\end{align}
\begin{align*}
    since \;\;\;  T = \frac{I_o}{I_s}
\end{align*}

\begin{align}
    \frac{I_o}I_s{}  = \frac{\mu g_{m2}R_s}{1 + \mu g_{m2}R_s}
    \label{eq_ee18btech11023_40}
\end{align}
\begin{align}
    I_o = I_s\sbrak{\frac{\mu g_{m2}R_s}{1 + \mu g_{m2}R_s}}
      \label{eq_ee18btech11023_41}
\end{align}
from eq.\ref{eq_ee18btech11023_35} substitute the value of $I_s$ in eq.\ref{eq_ee18btech11023_41}

\begin{align}
    I_o =  g_{m1} V_i\sbrak{\frac{\mu g_{m2}R_s}{1 + \mu g_{m2}R_s}}
    \label{eq_ee18btech11023_42}
\end{align}
\begin{align*}
    since\;\; \mu g_{m2}R_s >> 1
\end{align*}
\begin{align}
    I_o =  g_{m1} V_i\sbrak{\frac{\mu g_{m2}R_s}{ \mu g_{m2}R_s}}
    \label{eq_ee18btech11023_43}
\end{align}
\begin{align}
    I_o =  g_{m1} V_i
     \label{eq_ee18btech11023_44}
\end{align}
the expression for the output current is:
\begin{align}
    I_o =  g_{m1} V_i
     \label{eq_ee18btech11023_45}
\end{align}

the output amplifier is:
\begin{align}
    R_{out} = (1 + GH)R_o
    \label{eq_ee18btech11023_46}
\end{align}
from the question  Q.\ref{Question_1d_ee18btech11023} we will get:
\begin{align}
    GH = \mu g_{m2}R_s
    \label{eq_ee18btech11023_47}
\end{align}
\begin{align}
    R_{out} = (1 + \mu g_{m2}R_s)R_o
    \label{eq_ee18btech11023_48}
\end{align}

\begin{align}
    R_{out} \equiv \mu g_{m2}R_s R_o
     \label{eq_ee18btech11023_49}
\end{align}
therefore the expression for output resistance is :
\begin{align}
    R_{out} = \mu g_{m2}R_s R_o
     \label{eq_ee18btech11023_50}
\end{align}
\end{enumerate}
