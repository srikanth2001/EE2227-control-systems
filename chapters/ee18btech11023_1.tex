\begin{enumerate}[label=\thesection.\arabic*.,ref=\thesection.\theenumi]
\numberwithin{equation}{enumi}

\item
Using the Nyquist criterian , find out whether the system is stable or not 
\begin{align}
    G(s) = \frac{20}{s(s+1)}
\end{align}
\begin{align}
    H(s)=\frac{s+3}{s+4}
\end{align}

\item
\solution
\begin{align}
    G(s)H(s)=\frac{20(s+3)}{s(s+1)(s+4)}
\end{align}
\begin{align*}
     = \frac{20s+60}{s^{3}+5s^{2}+4s}
\end{align*}

\begin{align}
    1 + G(s)H(s)=\frac{s^{3}+5s^{2}+24s+60}{s^{3}+5s^{2}+4s}
\end{align}
\item
Nyquist Stability Criterion can be expressed as:
\begin{align}
    Z = N + P
\end{align}

Where:
    
\center Z = number of roots of 1+G(s)H(s) in right-hand side (RHS) of s-plane (It is also called zeros of characteristics equation)
\center N = number of encirclement of critical point 1+j0 in the clockwise direction
\center P = number of poles of open loop transfer function (OLTF) [i.e. G(s)H(s)] in RHS of s-plane.

\center Z=N+P is valid for all the systems whether stable or unstable. For the stable system, Z=0,
So for the stable system N = –P. 

\center  if p = 0 \;\;
\center there will be no  Encirclement of Nyquist plot and the system is stable
\begin{align}
    G(s)H(s)=\frac{20(s+3)}{s(s+1)(s+4)}
\end{align}
\begin{align}
    Here \, P = 0
\end{align}
\begin{align}
    Then\,  N = 0
\end{align} 
\includegraphics[width=10cm, height=10cm]{ee18btech11023_1.png}
\center   by seeing the we conclude that N = 0 and\;\;\; P = 0 
\begin{align}
    hence \;the\;systen \;is \;stable
\end{align}
\item
verify the answer with python code
https://github.com/srikanth2001/EE2227-control-systems/tree/master/codes
\end{enumerate}